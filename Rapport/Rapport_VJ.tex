\documentclass[a4paper, 12pt,twoside]{article}
\usepackage[utf8]{inputenc}
\usepackage[frenchb,noconfigs]{babel}
\usepackage[T1]{fontenc}
\usepackage[scaled]{beramono}
\renewcommand*\familydefault{ttdefault}
\usepackage{listings}
\lstset{
  language=[GNU]C++,
  showstringspaces=false,
  formfeed=\newpage,
  tabsize=4,
  commentstyle=\itshape,
  basicstyle=\ttfamily,
  morekeywords={models, lambda, forms}
}

\newcommand{\code}[2]{
  \hrulefill
  \subsection*{#1}
  \lstinputlisting{#2}
  \vspace{2em}
}

\usepackage{lmodern,textcomp,ifthen,graphicx,enumitem}
\usepackage[notes,
            titlepage,
            a4paper,
            pagenumber,
            sectionmark,
            twoside,
            fancysections]{polytechnique}
\usepackage[colorlinks=true,
            linkcolor=black,%bleu303,
            filecolor=red,
            urlcolor=bleu303,
            bookmarks=true,
            bookmarksopen=true]{hyperref}

\title{Détecteur de Viola Jones}
\subtitle{Projet \\ INF 442}
\author{
	Responsable~: J.-P. Bordes \\
        Felipe \textsc{Garcia} \\
        Francisco Eckhardt \\
        }
\date\today

\begin{document}
    \maketitle
    \renewcommand{\baselinestretch}{1.1}
    \setlength{\parskip}{0.5em}
    \tableofcontents
    \clearpage

%%%%%%%%%%%%%%%%%%%%%
%% DEBUT DU DOCUMENT %%%%%%
%%%%%%%%%%%%%%%%%%%%%
\section{Présentation du sujet}

\paragraph{Introduction} 
Qual é o objetivo: ...\\

\clearpage

%%%%%%%%%%%%%%%%%%%%%%%
%% EXPLICATION ALGORITHME %%%%%%
%%%%%%%%%%%%%%%%%%%%%%%
\section{Explication succincte du algorithme}

\subsection{L'algorithme}

\subsection{Calcul de l'Image Intégrale}
ref. Q.1.1

\subsection{Définition des caractéristiques en parallèle}
ref. Q1.2- commenter le nobre de caractéristiques ainsi obtenu

\subsection{Entraînement des classifieurs faibles en parallèle}
ref. Q.2.1, méthode Perceptron.
 Commenter le choix de $\epsilon$ et $K$. à quoi sert $\epsilon$ ? à quoi sert $K$ ?

\subsection{Boosting et choix de classifieurs}
ref. Q2.2 Commenter le choix de N dans AdaBoost

\clearpage

%%%%%%%%%%%%%%%%%%%%%%%
%% TEST DES RÉSULTATS %%%%%%
%%%%%%%%%%%%%%%%%%%%%%%

\section{Test et Analyse des résultats}

\subsection{Variation et analyse du paramètre $\theta \in [-1,1]$}
ref. Q.3.1. à quoi sert $\theta$ ?

\subsection{Description de l'extension de l’algorithme pour des images quelconques}
ref. Q.3.1

\clearpage

%%%%%%%%%%%%%%%%%
%% CONCLUSION? %%%%%%
%%%%%%%%%%%%%%%%%

%\section{Conclusion}
%
%Voici la conclusion de notre projet.
%
%
%%%%%%%%%%%%%%%%%%%%%%%%%
%%% LIGNE FINALE DE LA CONCLUSION %%
%%%%%%%%%%%%%%%%%%%%%%%%%
%\begin{center}
%\color{bleu303}
%\rule{0.3\textwidth}{0.2mm}\vspace*{-3.5mm}
%\rule{0.5\textwidth}{0.6mm}\vspace*{-3.8mm}
%\rule{0.3\textwidth}{0.2mm}\vspace*{-1mm}
%\sffamily FIN
%\end{center}
% \clearpage
 
%%%%%%%%%%%%%%
%%    BIBLIOGRAFIA %%
%%%%%%%%%%%%%%

\section{Références bibliographiques}
        {
        \renewcommand{\section}[2]{} % pour virer le titre "Références"
        \nocite{*} %% INCLUDE ALL BIB
        \bibliographystyle{plain}
        \bibliography{Rapport_VJ.bib}
        }
        
%%%%%%%%%%%%%%%%%%%%%
%% FIN DU DOCUMENT %%%%%%
%%%%%%%%%%%%%%%%%%%%%
    
\end{document}
