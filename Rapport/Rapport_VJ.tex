\documentclass[a4paper, 12pt,twoside]{article}
\usepackage[utf8]{inputenc}
\usepackage[frenchb,noconfigs]{babel}
\usepackage[T1]{fontenc}
\usepackage[scaled]{beramono}
\renewcommand*\familydefault{ttdefault}
\usepackage{listings}
\lstset{
  language=[GNU]C++,
  showstringspaces=false,
  formfeed=\newpage,
  tabsize=4,
  commentstyle=\itshape,
  basicstyle=\ttfamily,
  morekeywords={models, lambda, forms}
}

\newcommand{\code}[2]{
  \hrulefill
  \subsection*{#1}
  \lstinputlisting{#2}
  \vspace{2em}
}

\usepackage{lmodern,textcomp,ifthen,graphicx,enumitem}
\usepackage[notes,
            titlepage,
            a4paper,
            pagenumber,
            sectionmark,
            twoside,
            fancysections]{polytechnique}
\usepackage[colorlinks=true,
            linkcolor=black,%bleu303,
            filecolor=red,
            urlcolor=bleu303,
            bookmarks=true,
            bookmarksopen=true]{hyperref}

\title{Détecteur de Viola Jones}
\subtitle{Projet \\ INF 442}
\author{
	Responsable~: J.-P. Bordes \\
        Felipe \textsc{Garcia} \\
        Francisco Eckhardt \\
        }
\date\today

\begin{document}
    \maketitle
    \renewcommand{\baselinestretch}{1.1}
    \setlength{\parskip}{0.5em}
    \tableofcontents
    \clearpage

\section{Présentation du sujet}

\paragraph{Introduction} 
L'objectif de ce projet est: ...\\
Example de code:


%\begin{lstlisting}
%std::vector<std::vector<double> > IntegralImage(
%std::vector<std::vector<long> >& data) {
%	using namespace std;
%	int m = data.size();
%	int n = data[0].size();
%	
%  vector<vector<double> > output(m, vector<double>(n));
%  
%  output[0][0] = data[0][0];
%  
%  // initialize the first row
%  for(unsigned int i = 1; i < m; ++i)
%  {
%	  output[i][0] = data[i][0] + output[i-1][0];
%  }
%  // initialize the first column
%  for(unsigned int j= 1; j < n; ++j)
%  {
%	  output[0][j] = data[0][j] + output[0][j-1];
%  }
%  // Compute all the other values
%  for(unsigned int i = 1; i < m; ++i)
%  {
%	for(unsigned int j = 1; j < n; ++j)
%	{
%		output[i][j] = 
%		data[i][j] + output[i-1][j] + output[i][j-1] - output[i-1][j-1];
%	}
%  }
%
%	return output;
%}
%
%\end{lstlisting}
%% Use this to include files
%% \code{Models}{../testapp/models.py}


\paragraph{Aunautre paragraphe}

        ICI une extention du sujet

    \clearpage
    
\section{Q1.1}

        \subsection{Mise en place des moyens de communication}

            Nous avions décidé d'utiliser les moyens de communication suivants~:
           

    \clearpage
    \section{Q.2}

        Après révision des objectifs, voici ceux que nous souhaitons conserver pour cette deuxième partie de projet~:

        \begin{itemize}[label=\color{bleu303}\textbullet{}]
            \item Accélérer le programme pour le DNS poisoning
            \item Accélérer le programme TCP et anticiper sur les réponses suivantes
            \item Comment bloquer les attaques ci-dessus~?
        \end{itemize}

        \subsection{Accélérer le programme pour le DNS poisoning}

            kvejrkojvikre

        \subsection{Accélérer le programme TCP}% ne suis pas passé par ici

            lkfrfrf

        \subsection{Comment bloquer les attaques ci-dessus?}

            edokde

    \clearpage
   % \addcontentsline{toc}{section}{Conclusion}
    \section{Conclusion}

        Voici la conclusion de notre projet.

    \begin{center}
        \color{bleu303}

        \rule{0.3\textwidth}{0.2mm}\vspace*{-3.5mm}

        \rule{0.5\textwidth}{0.6mm}\vspace*{-3.8mm}

        \rule{0.3\textwidth}{0.2mm}\vspace*{-1mm}

        \sffamily FIN
    \end{center}
    
%    BIBLIOGRAFIA

    \clearpage
        \section{Références bibliographiques}
        {
        \renewcommand{\section}[2]{} % pour virer le titre "Références"
        \nocite{*}
        \bibliographystyle{plain}
        \bibliography{Rapport_VJ.bib}
        }

    
\end{document}
